\documentclass[14.5pt]{article}
\usepackage[utf8]{inputenc}
%\usepackage[spanish]{babel}
\usepackage{amsmath}
\usepackage{amsthm}
\usepackage{fancyhdr}
\usepackage{amsfonts}
\usepackage[margin=0.95in]{geometry}
\usepackage{comment}
\usepackage{xcolor}
\usepackage{mdframed}
\pagestyle{fancy}

\fancyhead[L]{Luis Gonzalez Rivas}
\fancyhead[R]{Real and Complex Analysis}

\newcommand{\N}{\mathbb{N}}
\newcommand{\Z}{\mathbb{Z}}
\newcommand{\Q}{\mathbb{Q}}
\newcommand{\R}{\mathbb{R}}

\newenvironment{pr}[2][Problem]{\begin{mdframed}[backgroundcolor=gray!10, leftline = false, rightline=false, linewidth=0.25pt]  \begin{trivlist}
\item[\hskip \labelsep {\bfseries #1}\hskip \labelsep {\bfseries #2.}]}{\end{trivlist} \end{mdframed}  }

\newenvironment{solution}
  {\begin{proof}[Solution]}
  {\end{proof}}
  
\begin{document}

\section{Abstract Integration}

\begin{pr}{1.1} Does there exist an infinite $\sigma$-algebra which has only countably many members?
\end{pr}
\begin{solution}
Let $X$ be any set and $S$ an infinite $\sigma$-algebra of $X$. Suppose that $S$ is countable. Then, we can write $S = \{A_i: i \in \N, A_i \subset X \}$. For each $x$ in $X$ define $B_x = \cap_{x \in A_i} A_i$. Notice that each $B_x$ is an element of $S$, since each one is a countable intersection of members of $S$. Let $x$ and $y$ two distinct elements of $X$, and suppose that $B_x \cap B_y \neq \emptyset.$ 
\end{solution}


%--------------------------------------
\begin{pr}{1.2} Prove an analogue of Theorem 1.8 for $n$ functions  
\end{pr}
\begin{proof}
Let us prove the statement by induction. Theorem 1.8 states that the case $n = 2$ holds. Let the statement be true all natural numbers less than $n$. Let $f_1, \ldots, f_n$ be measurable functions from $X$ to $\R$ and let $\Phi: \R^n \to Y$ be a continuous function. Notice that 
$$\Phi(f_1(x), \ldots, f_n(x)) = \Phi(I_{n-1}(f_1(x), \ldots, f_{n-1}(x)), f_n(x)),$$
where $I_{n-1}:\R^{n-1} \rightarrow \R^{n-1}$ is the identity in $\R^n$. In the standard topology, $I_{n-1}$ is continuous. Then, by induction, $I_{n-1}(f_1(x), \ldots, f_{n-1}(x))$ is a measurable function from $\R^{n-1} \rightarrow \R^{n-1}$. Hence, by induction, $\Phi(f_1(x), \ldots, f_n(x)) = \Phi(I_{n-1}(f_1(x), \ldots, f_{n-1}(x)), f_n(x))$
    
\end{proof}

%--------------------------------------
\begin{pr}{1.3} Prove that if $f$ is a real function on a measurable space $X$ such that $\{x:f(x) \geq r\}$ is measurable for every rational $r$, then $f$ is measurable.
\end{pr}
\begin{proof}
Let $\alpha \in \R$ and let $A = \{r \in \Q: r > \alpha\}.$ Notice that $(\alpha, \infty] = \bigcup_{r \in A} [r, \infty]$. Then, 
$$f^{-1}((\alpha, \infty]) = f^{-1}\left( \bigcup_{r \in A} [r, \infty] \right) = \bigcup_{r \in A} f^{-1}([r, \infty]).$$
Since $A$ is countable, the above equations implies that $f^{-1}((\alpha, \infty])$ is measurable. Given that $\alpha$ was chosen arbitrarily, $f$ is measurable.
\end{proof}

%--------------------------------------
\begin{pr}{1.4} Let $\{a_n\}$ and $\{b_n\}$ be sequences in $[-\infty, \infty]$, and prove the following assertions:
\begin{itemize}
    \item[(a)] $\limsup (-a_n) =  - \liminf a_n$.
    \item[(b)] $\limsup (a_n + b_n) \leq \limsup a_n  + \limsup b_n$
    provided that non of the sums is of the form $\infty - \infty$.
    \item[(c)] If $a_n \leq b_n,$ for all $n$, then
    $$\liminf a_n \leq \liminf b_n.$$
    Show by an example that strict inequality can hold in (b).
\end{itemize}
\end{pr}

\end{document}