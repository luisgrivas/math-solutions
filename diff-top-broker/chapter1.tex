\chapter{Manifolds and differentiable structures}

\begin{problem}{1}
Show that every (differentiable) manifold possesses a countable (differentiable) atlas.
\end{problem}
\begin{proof}
Let $\mathcal{A} = \{(U_\alpha, h_\alpha) \mid \alpha \in A \}$ be an atlas for $M.$ Since $M$ is second countable, there exists a countable basis $B = \{V_n \mid n \in \N \}.$ For each $p \in M$ there exist $V_k \in B$ and $(U_\beta, h_\beta) \in \mathcal A$ such that $p \in V_k \subset U_\beta.$ Since $B$ is countable, there is a sequence $\{\alpha_{k}\}_{k}^\infty \subset A$ such that $\bigcup_k U_{\alpha_k} = M$. It follows that $\{(U_{\alpha_k}), h_{\alpha_l} \mid k \in \N \}$ is a countable atlas of $M.$
\end{proof}

\begin{problem}{2}
Show that the sphere $S^n$ possesses a differentiable atlas with precisely two charts. Also, one with only one chart?
\end{problem}
\begin{proof}
Let $p_1 = (0, \ldots, 0, 1) \in S^n$ and $p_2 = (0,\ldots,0, -1) \in S^n$. Then the sets $U_ 1 = S^n - {p_1}$ and $U_2 = S^n - \{p_2\}$ are open. Define $h_1: U_1 \rightarrow \R^n$ and $h_2: U_2 \rightarrow \R^n$ by 
$$
h_1(x_1, \ldots, x_{n+1}) = \left(\frac{x_1}{1-x_{n+1}}, \ldots, \frac{x_n}{1-x_{n+1}}\right),
$$
$$
h_2(x_1, \ldots, x_{n+1}) = \left(\frac{x_1}{1+x_{n+1}}, \ldots, \frac{x_n}{1+x_{n+1}}\right).
$$

A direct argument shows that these functions are homeomorphisms from $U_1, U_2$ onto $\R^{n}.$ Their inverses, $h_1^{-1}, h_2^{-1}$, are defined in the following way:
$$
h_1^{-1}(x_1, \ldots, x_n) = \left( \frac{2x_1}{1 + r^2}, \ldots,  \frac{2 x_n}{1+r^2}, \frac{r^2-1}{1+r^2} \right),
$$
$$
h_2^{-1}(x_1, \ldots, x_n) = \left( \frac{2x_1}{1 - r^2}, \ldots,  \frac{2 x_n}{1-r^2}, \frac{r^2}{1-r^2} \right).
$$

with $r^2 = \sum_{k=1}^n x_k^2.$ Then $h_2 \circ h_1^{-1}: \R^n \rightarrow \R^n$ is defined 
$$ (h_2 \circ h_1^{-1}) (x_1, \ldots, x_n) = h_2 \left( \frac{2x_1}{1 + r^2}, \ldots,  \frac{2 x_n}{1+r^2}, \frac{r^2-1}{1+r^2} \right) = \frac{1}{r^2}(x_1, \ldots, x_n).$$

Hence $h_2 \circ h_1^{-1}$ is differentiable. In a similar way, $h_1 \circ h_2^{-1}$ is differentiable. Therefore, the atlas $\{(U_1, h_1), (U_2, h_2)\}$ defines a differentiable structure.

\end{proof}


\begin{problem}{3}
Describe the chart transformation for the atlas $\R P^n$ in (1.5(c)), and show that it is differentiable.
\end{problem}


\begin{problem}{4}
Let $M$ be a differentiable manifold and $\tau: M \rightarrow M$ a fixed point free involution, that is, $\tau$ is a diffeomorphism with $\tau \circ \tau = Id_M$ and $\tau(x) \neq x$ for all $x.$ 

Show that the quotient space $M/\tau$, which is obtained from $M$ by identification of points corresponding to each other under $\tau$, is a topological manifold which possesses exactly one differentiable structure with respect to which the projection $M \rightarrow M/\tau$ is locally diffeomorphic.
\end{problem}
\begin{proof}
Let $p: M \rightarrow M/\tau$ be the canonical projection and let $\mathcal A$ be a maximal atlas of $M$. Since $M$ is Hausdorff, for each $x \in M$ there exist open sets $U, V$ such that $x\in U$, $\tau(x) \in V$ and $U \cap V = \varnothing.$ Since $\tau$ is continuous, $W = U \cap \tau^{-1}(V)$ is a non-empty open set. This open set satisfies $W \cap \tau(W) = \varnothing$; then the set $\mathcal B = \{(W, k) \in \mathcal A \mid W \cap \tau(W) \}$ is non-empty.

For each $(W, k)\in \mathcal B$, the function $p$ is injective in $W$. Let $x, y \in W$ with $p(x) = p(y)$, that is, $x,y$ belong to the same equivalence class. But $W \cap \tau(W) = \varnothing$, hence $x=y.$ Since by definition $p$ is surjective, we have shown that $p$ is bijective in $W.$

A direct argument shows that the set $\mathcal C = \{(p(W), k \circ p^{-1}) \mid (W, k) \in \mathcal B \}$ induces a differentiable structure in $M/\tau$. To show that $p$ is locally diffeomorphic with respect to these differentiable structures, let $x \in M$ and $(W, k)\in \mathcal B$ with $x\in W$. The composition $(k \circ p^{-1}) \circ p \circ k^{-1} = Id_{\R^n}$ is differentiable. The inverse $p^{-1}: p(W) \rightarrow W$ of $p$ is also differentiable, since the composition $k \circ p^{-1}\circ (k \circ p^{-1})^{-1} = k \circ p^{-1} \circ (p \circ k^{-1}) = Id_{\R^n}$ is differentiable.

UNIQUENESS
\end{proof}

\begin{problem}{5}
Show that $\R P^1 \cong S^1.$
\end{problem}
\begin{proof}

\end{proof}

\begin{problem}{6}
Provide the surface of a cube $\{x\in \R^{n+1} \mid \max \{\lvert x_i \rvert\ = 1 \}$ with the structure of a differentiable manifold. 
\end{problem}

\begin{problem}{7}
Let $M$ be a differentiable manifold and $f: N \rightarrow M$ a homeomorphism. Prove that $N$ possesses exactly one structure as a differentiable manifold, so that $f$ is diffeomorphic.
\end{problem}

\begin{problem}{8}
Provide the complex projective sapce $\C P^n$ with the structure of a $2n-$dimensional differentiable manifold. This space is defined as follows: on the complex vector space $\C^{n+1},$ one has the equivalence relation $x \sim y$ if and only if there is a number $\lambda \in \C$, $\lambda \neq 0$, so that $\lambda x = y$. The quotient space $(\C - \{0\}) / \sim $ is defined to be $\C P^n$. 
\end{problem}

\begin{problem}{9}
Prove that if $M$ is a non-empty, $n-$dimensional manifold and $k \leq n$, then tere is an embedding $\R^k \rightarrow M.$
\end{problem}

\begin{problem}{10}
Let $N$ be a compact, $M$ a connected manifold, both of dimension $n$ and non-empty. Let $f: N \rightarrow M$ be an embedding. Show that $f$ is a diffeomorphism.
\end{problem}


\begin{problem}{11}
Show that $S^n$ is a submanifold of $\R^{n+1}.$
\end{problem}


\begin{problem}{12}
Describe an embedding $S^1 \times S^1 \rightarrow \R^3$ by means of elementary functions.
\end{problem}


\begin{problem}{13}
Show that the composition of two embeddings is again an embedding and the Cartesian product $f_1 \times f_2: N_1 \times N_2 \rightarrow M_1 \times M_2$, of two embeddings $f_1, f_2$, is again an embedding. 
\end{problem}


\begin{problem}{14}
Show that if the $n-$dimensional manifold $M$ is a product of spheres, then there exists an embedding $M\rightarrow \R^{n+1}.$
\end{problem}


\begin{problem}{15}
The points of $\C P^k$ are described by the homogeneous coordinates $x = [x_0, \ldots, x_k]:= \text{ class of} (x_0, \ldots, x_k)$ under $\sim$. Show that the mapping 
$$
f: \C P^m \times \C P^n \rightarrow \C P^{mn + m + n}
$$
$$
(x, y) \mapsto [x_0y_0, x_0 y_1, \ldots, x_v y_\mu, \ldots, x_m y_n ]
$$
is an embedding. Show the same for the real projective spaces. 
\end{problem}


\begin{problem}{16}
Let $M(m \times n)$ be the vector space of real $(m\times n)-$matrices, and $M_r(m \times n)$ the subset of matrices of rank $r$. Then $M_r(m \times n)$ is a submanifold of $M(m \times n)$ of codimension $(n-r)(m-r)$ for $r \leq \min \{m,n\}.$
\end{problem}


\begin{problem}{17}
The inclusion $\R^{n+1} \subset \R^{n+2}$ induces an embedding $\R P^n \subset \R P^n{n+1}$ and $\R P^{n+1} - \R P^{n} \cong \R^{n+1}.$
\end{problem}


\begin{problem}{18}
Let $\R^{n+1} = \{(x, a_0, \ldots, a_{n-1}) \mid x, a_i \in \R \}$. The set of points such that $x^n + a_{n-1}x^{n-1}+ \ldots + a_0 = 0$ is a submanifold of codimension $1$ of $\R^{n+1}$, and is diffeomorphic to $\R^n.$
\end{problem}


\begin{problem}{19}
The set $C^\infty(M)$ is an algebra under the natural addition and multiplication of functions. A differentiable mapping $f: M \rightarrow N$ defines an algebra homomorphism
$$
f^\ast: C^\infty(N) \rightarrow C^\infty(M), \ \ \phi \mapsto \phi \circ f
$$
with the functorial properties: $Id^\ast_M = Id;$ $(f \circ g)^\ast = g^\ast \circ f^\ast.$
\end{problem}

\begin{problem}{20}
Notation as in 19. For a point $p\in M$ let 
$$
\mathrm{M}_p = \{\phi \in C^\infty(M) \mid \phi(p) = 0 \}.
$$
Show:
\begin{itemize}
\item[(a)] $\mathrm{M}_p$ is a maximal ideal of $C^\infty(M).$
\item[(b)] If $M$ is compact and $\mathrm{M} \in C^\infty(M)$ is a maximal ideal, then tehere exists some $p \in \mathrm{M}$ such that $\mathrm{M} = \mathrm{M}_p.$
\end{itemize}
\end{problem}