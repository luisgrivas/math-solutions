\chapter{Tangent space}

\begin{problem}{1}
Show that $\mathfrak m(p):= \{\overline{\phi} \in \mathcal{E}(p) \mid \overline{\phi}(p) = 0 \}$ is the only maximal ideal of $\mathcal{E}(p).$
\end{problem}


\begin{problem}{2}
Show that if $p \in M^n$ and $n\neq 0$, then the ideal $\mathfrak m (p)$ in exercise 1 is not the only ideal $\neq 0$, $\mathcal{E}(p)$ of $\mathcal{E}(p).$
\end{problem}

\begin{problem}{3}
Show that if $f: M \rightarrow N$ is an embedding and $f(p) = q$, then the map $f^\ast: \mathcal{E}(q) \rightarrow \mathcal{E}(p)$ is surjective and $T_p(f)$ injective.
\end{problem}

\begin{problem}{4}
Show that the maximal ideal $\mathfrak m_n \subset \mathcal{E}_n$ is generated by the germs $\overline{x_1}, \ldots, \overline{x_n}$ of the coordinate functions. 
\end{problem}

\begin{problem}{5}
Show that if $\mathfrak m_n \subset \mathcal{E}_n$ is the maximal ideal, then $m_n^k$ is the ideal of the germs $\overline{f}$, for which all partial derivatives of order $<k$ vanish at the origin.
\end{problem}

\begin{problem}{6}
Show that the Taylor series at the point zero defines a homomorphism $\mathcal{E}_n \rightarrow \R[[x_1, \ldots x_n]]$ into the ring of formal power series in $n$ variables. The kernel of this homomorphism is $\mathfrak m_n^\infty:= \bigcap_{k=1}^\infty \mathfrak m_n^k$.
\end{problem}

\begin{problem}{7}

\end{problem}

\begin{problem}{8}

\end{problem}

\begin{problem}{9}

\end{problem}

\begin{problem}{10}

\end{problem}

\begin{problem}{1}

\end{problem}

\begin{problem}{1}

\end{problem}