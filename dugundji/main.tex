\documentclass[12pt]{article}
\usepackage[utf8]{inputenc}
%\usepackage[spanish]{babel}
\usepackage{amsmath}
\usepackage{amsthm}
\usepackage{amssymb}
\usepackage{fancyhdr}
\usepackage{mathpazo,amsfonts}
\usepackage[margin=0.95in]{geometry}
\usepackage{tikz}


\usepackage[
backend=biber,
style=alphabetic,
sorting=ynt
]{biblatex}

%\addbibresource{blb.bib}

%\pagestyle{fancy}


\newcommand{\N}{\mathbb{N}}
\newcommand{\Z}{\mathbb{Z}}
\newcommand{\Q}{\mathbb{Q}}
\newcommand{\R}{\mathbb{R}}
\newcommand{\E}{\mathbb{E}}


\author{Luis F. Gonz\'alez Rivas}
\title{Solutions to Dugundji's Topology}
\begin{document}
\maketitle
\section*{Chapter III. Topological Spaces}
\subsection*{Section I}

\newenvironment{problem}[2][Problema]{\begin{trivlist}
\item[\hskip \labelsep {\bfseries #1}\hskip \labelsep {\bfseries #2.}]}{\end{trivlist}}

\begin{problem}{5} In $\Z^+$, define $U \subset \Z^+$ to be open if it satisfies the condition: $n \in U \Rightarrow $ every divisor of $n$ belongs to $U$. Show that this is a topology in $\Z^+$ and it is not the discrete topology.  
\end{problem}

\begin{proof}
%Notice that $Z^+$ and $\emptyset$ are both open. Let $\{U_\alpha: \alpha \in \mathcal{A} \}$ be an arbitrary family of open sets and let $n \in U = \bigcup_{\alpha \in \mathcal{A}} U_\alpha$. Then, there is an $\beta \in \mathcal{A}$, such that $n \in U_\beta.$ Since $U_\beta$ is open, all divisors of $n$ are in $U_\beta$. Thus, all divisors of $n$ are in $U$. Therefore, $U$ is open. Now, let $\{V_i: i \in \Z^+\}$ be a numerable family of open sets, and let $m \in V = \bigcap_{i \in \Z^+} V_i.$ By definition, $m$ belongs to all sets $V_i$, $i \in \Z^+$. Thus, the divisors of $m$ are elements of every $V_i$, $i \in \Z^+$. Therefore $V$ is an open set.

Clearly this is not the discrete topology. For example, the set $\{6\}$ is not open.
\end{proof}

\begin{problem}{6} Prove: $\tau$ is the discrete topology if and only if every point is an open set.
\end{problem}
\begin{proof}
Let $\tau$ be the discrete topology of some nonempty set $X$. By definition, every subset of $X$ is open, in particular, every point is an open set. 

Suppose that every point is open and let $U$ be a subset of $X.$  Notice that $U = \bigcup_{u \in U} \{u\}$. Therefore, $U$ is open, since it is the union of open sets.
\end{proof}


\subsection*{Section 3}


\subsection*{Section 4}
\begin{problem}{1} Determine the closure, derived set, interior, and boundary, of the following sets: (a) The rationals in $\E$; (b) the Cantor set in $\E$; the set $\{(r_1, r_2): r_1, r_2 \in \Q \} \subset \E^2$; (d) $\{(x, 0): 0 < x < 1 \} \subset \E^2$.
\end{problem}
\begin{proof}
    \begin{itemize}
        \item[(a)]  
    \end{itemize}
\end{proof}
\end{document}