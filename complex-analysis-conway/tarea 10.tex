\documentclass[12pt]{article}
\usepackage[utf8]{inputenc}
\usepackage[spanish]{babel}
\usepackage{amsmath}
\usepackage{amsthm}
\usepackage{fancyhdr}
\usepackage{mathpazo,amsfonts}
\usepackage[margin=0.95in]{geometry}

\pagestyle{fancy}

\lhead{Tarea 10}
\chead{Luis González Rivas}
\rhead{22 de noviembre de 2021}


\usepackage[
backend=biber,
style=alphabetic,
sorting=ynt
]{biblatex}

\addbibresource{blb.bib}

\newtheorem{teo}{Teorema}


\newcommand{\N}{\mathbb{N}}
\newcommand{\Z}{\mathbb{Z}}
\newcommand{\Q}{\mathbb{Q}}
\newcommand{\R}{\mathbb{R}}
\newcommand{\C}{\mathbb{C}}

\newtheorem*{prop}{Proposición}

\newenvironment{problem}[2][Problema]{\begin{trivlist}
\item[\hskip \labelsep {\bfseries #1}\hskip \labelsep {\bfseries #2.}]}{\end{trivlist}}

\begin{document}
\section*{Análisis Complejo}

%-----------------------------------------
\begin{problem}{1. pp. 150} Demuestre el lema 1.5. 
\end{problem}
\begin{proof}
Sea $f(t) = \frac{t}{1+t}$ para $t > -1.$ Se comprueba que esta función es estrictamente creciente. Entonces, para $0 \leq t_0 \leq t_1 + t_2$, se tiene que $\frac{t_0}{1+t_0}\leq \frac{t_1}{1+t_1 +t_2} + \frac{t_2}{1+t_1+t_2} \leq  \frac{t_1}{1+t_1} + \frac{t_2}{1+t_2}.$

Si $\mu(s, t) = \frac{d(s,t)}{1+ d(s, t)}$, lo anterior muestra que $\mu$ satisface la desigualdad del triángulo. Además, es evidente que $\mu(s,t) = \mu(t,s)$ y que $\mu(s, t) = 0$ si y solo si $s =t.$ Por tanto $\mu$ es una métrica. 
\end{proof}
%-----------------------------------------

%-----------------------------------------
\begin{problem}{2. pp. 150} Determine los conjuntos $K_n$ obtenido de la proposición 1.2 para cada uno de los siguientes $G$: (a) $G$ es un disco abierto; (b) $G$ es un anillo abierto; (c) $G$ es el plano con $n$ discos cerrados disjuntos dos a dos removidos; (d) $G$ es una banda infinita; (e) $G = \C - \Z.$
\end{problem}
%-----------------------------------------

%-----------------------------------------
\begin{problem}{8. pp. 151} (a) Sea $f$ una función analítica en $B(0; R)$ y sea $f(z) = \sum_{n=0}^{\infty} a_n z^n$ para $\lvert z \rvert < R$. Si $f_n(z) = \sum_{k=0}^n a_k z^k$, demuestre que $f_n \to f$ en $C(G; \C).$

(b) Sea $G = ann(0; 0, R)$ y sea $f$ analítica en $G$ con serie de Laurent $f(z) = \sum_{n=-\infty}^\infty a_n z^n$. Sea $f_n(z) = \sum_{k=-\infty}^n a_k z^k$ y demuestre que $f_n \to f$ en $C(G; \C).$

\end{problem}
%-----------------------------------------

%-----------------------------------------
\begin{problem}{10 pp. 154} Sea $\{f_n\} \subset H(G)$ una sucesión de funciones uno a uno que converge a $f$. Demuestre que $f$ es uno a uno o $f$ es constante.
\end{problem}
%-----------------------------------------


%-----------------------------------------
\begin{problem}{5. pp. 173} Determine la convergencia del producto infinito $\prod_{n=1}^\infty \frac{1}{n^p}$ para $p > 0.$
\end{problem}
%-----------------------------------------

%-----------------------------------------
\begin{problem}{6. pp. 173} Determine la convergencia de los productos $\prod \left[ 1 + \frac{i}{n}\right]$ y $\prod \left|1 + \frac{i}{n} \right|$.
\end{problem}
%-----------------------------------------

%-----------------------------------------
\begin{problem}{7. pp. 173} Demuestre que $\prod_{n=2}^\infty \left(1-\frac{1}{n^2}\right) = \frac{1}{2}.$

\end{problem}
%-----------------------------------------
\printbibliography


\end{document}