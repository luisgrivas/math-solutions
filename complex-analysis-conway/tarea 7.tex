\documentclass[12pt]{article}
\usepackage[utf8]{inputenc}
\usepackage[spanish]{babel}
\usepackage{amsmath}
\usepackage{amsthm}
\usepackage{fancyhdr}
\usepackage{mathpazo,amsfonts}
\usepackage[margin=0.95in]{geometry}

\pagestyle{fancy}

\lhead{Tarea 7}
\chead{Luis González Rivas}
\rhead{12 de octubre de 2021}


\usepackage[
backend=biber,
style=alphabetic,
sorting=ynt
]{biblatex}

\addbibresource{blb.bib}



\newcommand{\N}{\mathbb{N}}
\newcommand{\Z}{\mathbb{Z}}
\newcommand{\Q}{\mathbb{Q}}
\newcommand{\R}{\mathbb{R}}
\newcommand{\C}{\mathbb{C}}

\newtheorem*{prop}{Proposición}

\newenvironment{problem}[2][Problema]{\begin{trivlist}
\item[\hskip \labelsep {\bfseries #1}\hskip \labelsep {\bfseries #2.}]}{\end{trivlist}}

\begin{document}
\section*{Análisis Complejo}

%---------------------------------
\begin{problem}{1}
Calcule las siguientes integrales: \text{ }
\begin{itemize}
    \item[(a)] $\int_{0}^\infty \frac{x^2 dx}{x^4 + x^2 + 1}$,
    \item[(b)] $\int_{0}^\infty \frac{\cos x - 1}{x^2} dx$
\end{itemize}
\end{problem}
\textit{Solución.} \text{ }
\begin{itemize}
    \item[(a)] Consideremos la función $f(z) = \frac{z^2}{z^4 + z^2 + 1}$. Esta se puede expresar de la siguiente manera: 
    $$f(z) = \frac{z^2}{(z-\alpha_1) (z-\alpha_2) (z - \overline{\alpha_1}) (z - \overline{\alpha_2}) },$$
    con $\alpha_1 = \frac{1}{2} + \frac{\sqrt{3}}{2} i$ y $\alpha_2 = -\frac{1}{2} + \frac{\sqrt{3}}{2}.$ Observe que $\alpha_1$ y $\alpha_2$ son polos de orden $1$. Un cálculo directo muestra que
    
    $$ Res(f,\alpha_1) = \frac{\alpha_1^2}{(\alpha_1 - \alpha_2)(\alpha_1 - \overline{\alpha_1})(\alpha_1 - \overline{\alpha_2})} = \frac{-1 + \sqrt{3} i}{2(-3 + \sqrt{3} i)},$$
    
    $$ Res(f, \alpha_2) = \frac{\alpha_2^2}{(\alpha_2 - \alpha_1) (\alpha_2 - \overline{\alpha_1})(\alpha_2 - \overline{\alpha_2})} = \frac{1 + \sqrt{3} i}{ -2 (3 + \sqrt{3} i)}. $$
    
    Por el teorema del residuo se tiene que 
    
    $$\frac{1}{2\pi i} \int_\gamma f = Res(f; \alpha_1) + Res(f; \alpha_2) = \frac{- \sqrt{3} i}{6},$$
    donde $\gamma$ es la curva compuesta por las curvas $\gamma_1(t) = R e^{ i t}$, $0 \leq t \leq \pi$, y $\gamma_2(t) = t$, con $-R \leq t \leq R$ y $R > 1.$ A su vez, 
    \begin{eqnarray*}
    \frac{1}{2 \pi i} \int_\gamma f &=& \frac{1}{2 \pi i} \left( \int_{\gamma_1} f + \int_{\gamma_2} f \right) \\
    &=& \frac{1}{2 \pi i} \left( \int_0^\pi \frac{R^3 i e^{3it}}{R^4 e^{4it} + R^2e^{2it} + 1} dt + \int_{-R}^R \frac{x^2}{x^4 + x^2 +1 } dx \right) \\
    &=& \frac{1}{2 \pi i} \left( \int_0^\pi \frac{R^3 i e^{3it}}{R^4 e^{4it} + R^2e^{2it} + 1} dt + 2 \int_{0}^R \frac{x^2}{x^4 + x^2 +1 } dx \right).
    \end{eqnarray*}
    El primer término de la expresión anterior tiende a cero cuando $R$ tiende a infinito. Luego,
    \begin{eqnarray*}
    \int_0^\infty \frac{x^2}{x^4 + x^2 +1} dx = \lim_{R \to \infty} \int_0^R \frac{x^2}{x^4 + x^2 +1} dx = \pi i \cdot \frac{-\sqrt{3} i}{6} = \frac{ \pi \sqrt{3}}{6}.
    \end{eqnarray*}
    
    \item[(b)] Consideremos la función $f(z) = \frac{e^{iz}-1}{z^2}$. Por el teorema de Cauchy, $\int_\gamma f = 0$, donde $\gamma$ es la curva compuesta por las curvas $\gamma_1(t) = R e^{it}$, con $0 \leq t \leq \pi $, $\gamma_2(t) = t$, con $-R \leq t \leq r,$ $\gamma_3(t) = r e^{it}$, con $-\pi \leq t \leq 0$,  $\gamma_4(t) = t,$ con $r \leq t \leq R$ y $0 < r < R.$ Luego, 
    
    \begin{eqnarray*}
    0 &=& \int_{\gamma_1} f + \int_{\gamma_2} f + \int_{\gamma_3} f + \int_{\gamma_4} f \\
    &=& \int_{\gamma_1} \frac{e^{iz}-1}{z^2} dz  + \int_{-R}^{-r} \frac{e^{ix}-1}{x^2}dx + \int_{\gamma_3} \frac{e^{iz}-1}{z^2} dz + \int_{r}^R \frac{e^{ix}-1}{x^2}dx  \\
    &=& \int_0^\pi \frac{i(e^{iR e^{it}}-1)}{R e^{it}} dt + \int_r^R \frac{e^{ix} + e^{-ix}-2}{x^2}dx + \int_\pi^{0} \frac{i (e^{i r e^{it}}-1)}{r e^{it}} dt\\
    &=&  \int_0^\pi \frac{i(e^{iR e^{it}}-1)}{R e^{it}} dt +  \int_\pi^{0} \frac{i (e^{i r e^{it}}-1)}{r e^{it}} dt + 2 \int_r^R \frac{\cos x - 1}{x^2}dx.
    \end{eqnarray*}
    
    Véase que 
    
    \begin{eqnarray*}
    \left\lvert \int_0^\pi \frac{i(e^{iR e^{it}}-1)}{R e^{it}} dt \right\rvert &\leq& \frac{1}{R} \int_0^\pi{\left\lvert \frac{e^{i R e^{it} }-1}{e^{it}} \right\rvert} dt\\
    &\leq& \frac{1}{R} \left( \int_0^\pi\lvert \exp{(iRe^{it})} \rvert dt + \int_0^\pi dt \right) \\
    &=& \frac{1}{R} \left(\int_0^\pi \exp{(- R \sin t)} dt + \pi \right).
    \end{eqnarray*}
    
    Esta última expresión tiende a cero cuando $R$ tiende a infinito. Por otro lado, $\int_{\gamma_3} \frac{e^{iz}-1}{z^2} dz = 2\pi i \cdot  -i/2 =  \pi$, para toda $r > 0$. Por tanto, 
    
    $$\int_{0}^\infty \frac{\cos x - 1}{x^2} dx = \lim_{R \to \infty } \lim_{r \to 0} \int_{r}^R \frac{\cos x -1 }{x^2}dx = - \frac{\pi}{2}.$$

\end{itemize}
%---------------------------------

%---------------------------------
\begin{problem}{2}
Verifique las siguientes ecuaciones: \text{ }
\begin{itemize}
    \item[(a)] $\int_0^{\infty} \frac{dx}{(x^2 + a^2)^2} = \frac{\pi}{4a^3},$ $a > 0;$
    \item[(c)] $\int_0^\infty \frac{\cos ax }{(1+x^2)^2} dx = \frac{\pi (a + 1)e^{-a}}{4}$, si $a>0;$
    \item[(d)] $\int_0^\frac{\pi}{2} \frac{d \theta}{a + \sin^2 \theta} = \frac{\pi}{2[a (a+1)]^{\frac{1}{2}}},$ si $a > 0;$
    \item[(e)] $\int_0^\infty \frac{\log x}{(1+x^2)^2} dx = -\frac{\pi}{4};$
    \item[(f)] $\int_0^\infty \frac{dx}{1+x^2} = \frac{\pi}{2};$
    \item[(g)] $\int_{-\infty}^\infty \frac{e^{ax}}{1+e^x} dx = \frac{\pi}{\sin a \pi}$ si $0 < a < 1;$
\end{itemize}
\end{problem}
\begin{proof} \text{ }
\begin{itemize}
\item[(a)] Considere la función $f(z) = \frac{1}{(z^2 + a^2)^2}$. Esta función tiene dos polos de orden dos en $z = ia$ y en $z = -ia$. Sea $\gamma$ la curva cerrada compuesta por las curvas $\gamma_1(t) = R e^{it}$, con $0 \leq t \leq \pi$, y $\gamma_2(t) = t$, con $- R \leq t \leq R$,  y $R > a.$ Entonces
\begin{eqnarray*}
\int_\gamma f &=& \int_{\gamma_1} f + \int_{\gamma_2} f \\
&=& \int_0^\pi \frac{iR e^{it}}{(R^2 e^{2it} + a^2 )^2}dt + \int_{-R}^R \frac{1}{(x^2 + a^2)^2}dx \\
&=&  \int_0^\pi \frac{iR e^{it}}{(R^2 e^{2it} + a^2 )^2} dt + 2 \int_{0}^R \frac{1}{(x^2 + a^2)^2} dx.
\end{eqnarray*}

Observe que el integrando del primer término del lado derecho tiende a cero cuando $R$ tiende a infinito. Sea $g(z) = (z - ia)^2 f(z) = \frac{1}{(z+ia)^2}$. Luego, $Res(f, ia) = g^\prime(ia) =  \frac{-2}{(2ia)^3} = \frac{1}{ 4a^3 i}$. 
Por el teorema del residuo $\int_\gamma f = 2 \pi i Res(f, ia) = \frac{\pi}{2a^3}.$ 

$$\int_0^\infty \frac{1}{(x^2 + a^2)^2} dt= \lim_{R \to \infty} \int_0^R \frac{1}{(x^2 + a^2)^2} dt = \frac{\pi}{4 a^3.} $$

\item[(c)] Considere la función $f(z) = \frac{e^{iaz}}{(1 + z^2)^2}$. Esta función tiene dos polos de orden dos en $z = i$ y en $z = -i$. Sea $g(z) = (z-i)^2 f(z) = \frac{e^{iaz}}{(z+i)^2}$. Entonces $Res(f, i) = g^\prime(i) = ia e^{-a}(2i)^{-2} -2(2i)^{-3} e^{-a} = \frac{iae^{-a}}{-4} + \frac{2e^{-a}}{8i}  = \frac{e^{-a}(1+a)}{4i}.$

Sea $\gamma$ la curva cerrada compuesta por las curvas $\gamma_1(t) = R e^{it}$, con $0 \leq t \leq \pi$, y $\gamma_2(t) = t$, con $- R \leq t \leq R$,  y $R > 1.$ Entonces

\begin{eqnarray*}
\int_\gamma f &=& \int_{\gamma_1} f + \int_{\gamma_2} f \\
&=& \int_0^\pi \frac{\exp{(ia R e^{it})} Ri }{(1 + R^2 e^{2it})^2} dt + \int_{-R}^R \frac{e^{iax}}{(1+x^2)^2} dx \\
&=& \int_0^\pi \frac{\exp{(ia R e^{it})} Ri }{(1 + R^2 e^{2it})^2} dt + \int_0^R \frac{e^{iax} + e^{-iax}}{(1+x^2)^2} dx \\
&=&  \int_0^\pi \frac{\exp{(ia R e^{it})} Ri }{(1 + R^2 e^{2it})^2} dt + 2 \int_0^R \frac{\cos x}{(1 +x^2)^2}dx. 
\end{eqnarray*}

Entonces, por el teorema del residuo, 
$$ \int_0^\infty \frac{\cos x}{(1 +x^2)^2}dx = \lim_{R \to \infty} \int_0^R \frac{\cos x}{(1+x^2)^2} dx = (\pi i) \cdot \frac{e^{-a}(1+a)}{4i} = \frac{\pi e^{-a}(1+a)}{4}.$$

\item[(d)] Sea $C$ el círculo unitario centrado en el origen. Observe que, si $z \in C$, entonces $z = e^{i \theta}$ y $\overline{z} = 1/z$. Entonces

$$a+ \sin^2(\theta) = a + (z - \overline{z})^2/-4 = - \frac{z^4 - (4a+2)z^2 + 1}{4z^2} = - \frac{(z^2 - \alpha)(z^2-\beta}{4z^2},$$
con $\alpha = 2a + 1 + 2 \sqrt{a(a+1)}$ y $\beta = \overline{\alpha}.$ Es claro que $\lvert \alpha \rvert > 1$. Por otro lado, por la desigualdad media geométrica  media aritmética, $\beta > 0$. Además, $\beta^2 = (2a+1- 2 \sqrt{a(a+1)})^2 = 1 - 4(2a+1)\sqrt{a(a+1)}$, por lo que $\beta < 1.$ Por tanto $\beta$ está en el interior de $C.$

Si $w = z^2$, entonces

\begin{eqnarray*}
\int_0^{\pi/2} \frac{d\theta}{a + \sin^2(\theta)} &=& \frac{1}{2}\int_0^{\pi} \frac{d\theta}{a + \sin^2(\theta)}\\
&=& - \frac{2}{i} \int_0^{\pi} \frac{z dz}{(z^2 - \alpha)(z^2-\beta)}\\
&=& -\frac{1}{i} \int_0^{2 \pi} \frac{dw}{(w-\alpha)(w - \beta)}\\
&=&  - \frac{1}{i} \cdot 2 \pi i \cdot Res(f, \beta) \\
&=& \frac{\pi}{2 \sqrt{a(a+1)}}.
\end{eqnarray*}

\item[(e)] Tomaremos como cierto que  $\int_0^\infty \frac{dx}{(1+x^2)^2} = \frac{\pi}{4}$. Sea $f(z) = \frac{l(z)}{(z^2 +1)^2}$, con $l(z) = \log \lvert z \rvert + i \theta$ y $- \frac{\pi}{2} < \theta < \frac{3\pi}{2}$. Esta función tiene dos polos de orden dos en $z=i$ y en $z=-i.$ Si $g(z) = (z-i)^2 f(z)$, entonces

$$Res(f, i) = g^\prime(i) = \frac{i\pi - 2}{8i}.$$

Sea $\gamma$ es la curva compuesta por las curvas $\gamma_1(t) = R e^{it}$, con $0 \leq t \leq \pi$, y $\gamma_2(t) = t$, con $- R \leq t \leq -r < -1$, $\gamma_3(t) = r e^{it}$, con $\pi \leq t \leq 2\pi$, y $\gamma_4(t) = t$, con $1 < r\leq t \leq R.$ Luego,

\begin{eqnarray*}
\int_\gamma f &=& \int_r^R \frac{\log x}{(1+x^2)^2}dx + iR \int_0^\pi \frac{l(Re^{it}) e^{it}}{(1 + R^2 e^{2it})^2} dt\\ &+& \int_{-R}^{-r} \frac{\log \lvert x \rvert }{(1+x^2)^2} dx + i \pi \int_{-R}^{-r} \frac{dx}{(1+x^2)^2} + ir \int_{\pi}^0 \frac{l(re^{it}) e^{it}}{(1+r^2 e^{2it})^2}dt.
\end{eqnarray*}

Si $r \to 0$, entonces $ir \int_{\pi}^0 \frac{l(re^{it}) e^{it}}{(1+r^2 e^{2it})^2}dt \to 0$; y si $R \to \infty$, entonces $iR \int_0^\pi \frac{l(Re^{it}) e^{it}}{(1 + R^2 e^{2it})^2} dt \to 0$. Luego,

$$
\int_0^\infty \frac{\log x}{(1+x^2)^2} dx =  \pi i \left( \frac{\pi i -2}{8i} - \frac{i \pi^2}{4} \right) = - \frac{\pi}{2}
$$


\item[(f)] Considere la función $f(z) = \frac{1}{1 + z^2}$. Este tiene dos polos simples en $z = i$ y en $z = i.$ Sea $\gamma$ la curva cerrada compuesta por las curvas $\gamma_1(t) = R e^{it}$, con $0 \leq t \leq \pi$, y $\gamma_2(t) = t$, con $- R \leq t \leq R$,  y $R > 1.$ Entonces

\begin{eqnarray*}
\int_\gamma f &=& \int_{\gamma_1} f + \int_{\gamma_2} f \\
&=& \int_0^\pi \frac{Ri e^{it}}{1 + R^2 e^{2it}} dt + \int_{-R}^R \frac{1}{1 + x^2} dx \\
&=& \int_0^\pi \frac{Ri e^{it}}{1 + R^2 e^{2it}} dt + 2 \int_0^R \frac{1}{1 + x^2} dx.
\end{eqnarray*}
Observe que el integrando del primer término del lado derecho tiende a cero cuando $R$ tiende a infinito. Sea $g(z) = (z-i)f(z) = \frac{1}{z+i}$. Entonces el $Res(f, i) = \frac{1}{2i}.$ Por el teorema del residuo,

$$\int_0^\infty \frac{1}{1 + x^2} dx = \lim_{R \to \infty} \int_0^R \frac{1}{1 + x^2} dx = \pi i \frac{1}{2i} = \frac{\pi}{2}. $$

\item[(g)] Considere la función $f(z) = \frac{e^{az}}{1+e^z}$. Observe que $f$ tiene un polo en $z=\pi i$. Para calcular el residuo en $\pi i$, observe que 

$$ (z - \pi i) f(z) = e^{az} \frac{z - \pi i}{1 + e^z} = e^{az} \frac{z- \pi i}{e^z - e^{\pi i}},$$

por lo que  $Res(f, \pi i) = \lim_{z \to \pi i} (z - \pi i) f(z) = - e^{a \pi i}.$

Sea $\gamma$ la curva rectangular definida por las curvas $\gamma_1(t) = R + it$, con $0 \leq t \leq 2 \pi i$, $\gamma_2(t) = -t + 2 \pi i$, $-R \leq t \leq R$, $\gamma_3(t) = -R + it$, con $0 \leq t \leq 2\pi$ y $\gamma_4(t) = t$, con $-R \leq t \leq R.$ Observe que el punto $z=\pi i$ se encuentra dentro de la curva $\gamma$, por lo que 

$$\int_\gamma f = -2\pi i e^{a \pi i}.$$

Ahora bien, tenemos que 

\begin{eqnarray*}
\int_{\gamma}f &=& \int_{\gamma_1} f - \int_{-R}^R \frac{e^{a(t + 2 \pi i)}}{1+e^{t + 2\pi i }}dt + \int_{\gamma_3} f + \int_{-R}^R \frac{e^{ax}}{1+ e^x} dx\\
&=& \int_{\gamma_1} f - e^{2a\pi i} \int_{-R}^R \frac{e^{a x}}{1+e^{x }}dx + \int_{\gamma_3} f + \int_{-R}^R \frac{e^{ax}}{1+ e^x}dx\\
&=& \int_{\gamma_1} f + \int_{\gamma_3} f + (1 - e^{2a\pi i})\int_{-R}^R \frac{e^{a x}}{1+e^{x }}dx.
\end{eqnarray*}

A su vez,

$$ \left\lvert \int_{\gamma_1} f \right\rvert = \left\lvert \int_{0}^{2 \pi}\frac{e^{a(R + it)}}{1+e^{R+it}}dt \right\rvert 
\leq \int_0^{2\pi}\left\lvert \frac{e^{a(R + it)}}{1+e^{R+it}}\right\rvert dt \leq 2\pi  e^{R(a-1)}.$$

Como $0 < a < 1$, el lado derecho de la desigualdad anterior tiende a cero cuando $R$ tiende a infinito. Por lo que  $\int_{\gamma_1} f \to 0$ cuando $R \to \infty.$ De manera similar se demuestra que la integral $\int_{\gamma_3} f$ tiende a cero cuando $R$ tiende a infinito. Con lo anterior, tenemos que

$$\int_{-\infty}^\infty \frac{e^{ax}}{1+e^x}dx = \lim_{R \to \infty} \int_{-R}^R  \frac{e^{ax}}{1+e^x}dx = \frac{-2 \pi i e^{a \pi i}}{1-e^{2 a \pi i}} = - \frac{2 \pi i}{e^{-a \pi i} - e^{a \pi i}} = \frac{\pi}{\sin(a \pi )}$$
\end{itemize}
\end{proof}

%---------------------------------


%---------------------------------
\begin{problem}{6} 
Sea $\gamma$ el camino rectangular $[n+\frac{1}{2} + ni, -n -\frac{1}{2} + ni, -n - \frac{1}{2} -ni, n+\frac{1}{2}-ni, n + \frac{1}{2}+ni]$ y evalue la integral $\int_\gamma \pi(z + a )^{-2} \cot \pi z dz$ para $a \neq$ entero. Demuestre que $\lim_{n \to \infty} \int_\gamma \pi(z+a)^{-2} \cot \pi z dz = 0$ y, utilizando la primera parte, deduzca que 
$$\frac{\pi}{\sin^2 \pi a} = \sum_{n=-\infty}^\infty \frac{1}{(a+n)^2}.$$
\end{problem}
\textit{Solución.} Sea $f(z) = \frac{\pi \cot(\pi z)}{(z+a)^2}$. Esta tiene polos simples en $z \in \Z$ y un polo de orden dos en $z=-a.$ Si $g(z) = (z+a)^2 f(z)$, el residuo de $f$ en $z=-a$ es

$$Res(f, -a) = g^\prime(-a) = \frac{- \pi^2 }{\sin^2(-\pi a)} = \frac{- \pi^2 }{\sin^2(\pi a)} .$$

Por otro lado, si $z = k \in \mathbb Z$, entonces

\begin{eqnarray*}
Res(f, k) &=& \lim_{z \to k} (z-k) f(z)\\
&=& \lim_{z \to k} \frac{\pi (z-k) \cot(\pi z) }{(z+a)^2} \\
&=& \lim_{z \to k} \frac{i \pi (z-k)(e^{i \pi z} + e^{- i \pi z}) }{(z+a)^2 (e^{i \pi z}-e^{-i \pi z})}\\
&=& \lim_{z \to k} \frac{i \pi (z-k)(e^{2 i \pi z} + 1) }{(z+a)^2 (e^{2i \pi z}-1)}\\
&=& i \pi \lim_{z \to k} \frac{(e^{2 i \pi z} + 1) }{(z+a)^2} \cdot  \lim_{z \to k} \frac{z-k}{e^{2 i \pi z }-e^{2 i \pi k}}\\
&=& i \pi \cdot \frac{2}{(k+a)^2} \cdot \frac{1}{2i} \\
&=& \frac{\pi}{(k+a)^2}.
\end{eqnarray*}

Observe también, para $n > 0$, los polos $z= 0, \pm 1, \ldots, \pm n$, están dentro de la curva $\gamma.$ Así, por el teorema del residuo 

$$ \frac{1}{2 \pi i} \int_{\gamma} f = \sum_{k=-n}^n \frac{\pi}{(k+a)^2} - \eta(\gamma, -a) \frac{\pi^2}{\sin^2(\pi a)}.$$

Por otro lado, observe que $\lvert \cot(\pi z) \rvert^2 = \frac{\cos^2(\pi x) + \sinh^2(\pi y)}{\sin^2(\pi x) + \sinh^2(\pi y)}$, si $z = x+iy$. Si consideramos el segmento vertical de $\gamma$, $n+\frac{1}{2}+ti$, con $-n \leq t \leq n$, entonces 

$$\lvert \cot(\pi z)\rvert^2 = \frac{\cos^2(n \pi + \frac{\pi}{2}) + \sinh^2(\pi t)}{\sin^2(n\pi + \frac{\pi}{2}) + \sinh^2(\pi t)} = \frac{\sinh^2(\pi t)}{1+\sinh^2(\pi t)}< 1.$$

Si consideramos el segmento horizontal de la curva $\gamma$, $-t+ni$, con $-n -\frac{1}{2}\leq t \leq n + \frac{1}{2}$, entonces

$$\lvert \cot(\pi z)\rvert^2 = \frac{\cos^2(-t \pi) + \sinh^2(\pi n)}{\sin^2(-t\pi) + \sinh^2(\pi n)} < 2,$$

para $n$ suficientemente grande. De manera similar, se siguen las mismas desigualdades en cada uno de los segmentos de $\gamma.$ Entonces

\begin{eqnarray*}
\left\lvert \int_{\gamma} f \right\rvert &\leq& \int_{\gamma} \left\lvert \frac{\pi \cot(\pi z)}{(z+a)^2} \right\rvert dz\\ &<& 2\pi \int_\gamma \frac{dt}{(z+a)^2}\\
&=& \frac{4in\pi}{(n+a +\frac{1}{2})^2 + n^2} - \frac{4in\pi}{(a -n -\frac{1}{2})^2 + n^2}.
\end{eqnarray*}
Cuando $n \to \infty$, el último término de la expresión anterior tiende a cero. Por tanto, $\int_\gamma \frac{\pi \cot(\pi z)}{(z+a)^2}dz \to 0$, cuando $n \to \infty$. Luego,

$$\sum_{n=-\infty}^\infty \frac{\pi}{(n+a)^2} = \frac{\pi^2}{\sin^2(\pi a)}. $$


%---------------------------------

%---------------------------------
\begin{problem}{7}
Utilice el ejercicio 6 para deducir que 
$$\frac{\pi^2}{8} = \sum_{n=0}^\infty \frac{1}{(2n + 1)^2} $$
\end{problem}
\textit{Solución.} Si $a=1/2$, entonces $\sin^2(\pi/2) = 1$ Luego, por el problema anterior,
$$\pi^2 = \sum_{n=-\infty}^\infty \frac{1}{(1/2 + n)^2} = \sum_{n=-\infty}^\infty \frac{4}{(1+ 2n)^2} = 8 \sum_{n=0}^\infty \frac{1}{(1 + 2n)^2}.$$
La expresión se sigue inmediatamente.
%---------------------------------
\printbibliography

\end{document}